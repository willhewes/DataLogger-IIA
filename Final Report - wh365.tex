\documentclass[a4paper,11pt]{article}
% Use ctrl + alt + V to view live pdf

% Packages
\usepackage[utf8]{inputenc} % For encoding
\usepackage[T1]{fontenc} % Better handling of accented characters and hyphenation
\usepackage{microtype} % Improves spacing and justification
\usepackage{amsmath, amssymb} % For equations and symbols
\usepackage{graphicx} % For including graphics/images
\usepackage{caption} % For customizing figure and table captions
\usepackage{subcaption} % For subfigures and subcaptions
\usepackage{float} % For fixing figure and table positions
\usepackage{booktabs} % For professional-looking tables
\usepackage{siunitx} % For consistent typesetting of units and numbers
\usepackage[margin=2cm]{geometry} % Adjusts page margins
\usepackage{fancyhdr} % For custom headers and footers
\usepackage{lmodern} % For a professional-looking font (main body font)
\usepackage{titlesec} % For title customization
\usepackage{array} % For custom table formatting
\usepackage[colorlinks=true, linkcolor=black, urlcolor=black]{hyperref} % Colored links without boxes
\usepackage{cleveref} % For improved cross-referencing    
\usepackage{multirow}
\usepackage{enumitem}
\usepackage{listings}
\usepackage{xcolor}
\usepackage{textcomp}
\usepackage{tabularx}
\usepackage{changepage}
\usepackage{tikz}
\usepackage{pdfpages}
\usepackage[table]{xcolor}
\usetikzlibrary{shapes.geometric, arrows}
% --- C++ Style ---
\lstdefinestyle{cpp-style}{
    language=C++,
    basicstyle=\ttfamily\footnotesize,
    keywordstyle=\color{blue}\bfseries,
    stringstyle=\color{orange},
    commentstyle=\color{gray}\itshape,
    numbers=left,
    numberstyle=\tiny\color{gray},
    numbersep=10pt,
    backgroundcolor=\color{white},
    showspaces=false,
    showstringspaces=false,
    breaklines=true,
    breakatwhitespace=true,
    tabsize=4,
    captionpos=b,
    frame=single,
    rulecolor=\color{black},
}

% --- Python Style ---
\lstdefinestyle{python-style}{
    language=Python,
    basicstyle=\ttfamily\footnotesize,
    keywordstyle=\color{blue}\bfseries,
    commentstyle=\color{gray}\itshape,
    stringstyle=\color{green!50!black},
    frame=single,
    breaklines=true,
    showstringspaces=false,
    captionpos=b
}
\renewcommand{\lstlistingname}{Code}
% Custom settings
\pagestyle{fancy}
\fancyhf{}
\fancyhead[L]{\textit{SF4 - DataLogger}} % Header left
\fancyhead[R]{\textit{Will Hewes - wh365}} % Header right 
\fancyfoot[C]{\thepage} % Footer center
\setlength{\headheight}{15pt} % Header height
\setlength{\parindent}{0em} % Indentation for paragraphs
\setlength{\parskip}{0.5em} % Add spacing between paragraphs
\setlength{\abovedisplayskip}{1em}
\setlength{\belowdisplayskip}{1em}
\setlength{\abovedisplayshortskip}{1em}
\setlength{\belowdisplayshortskip}{1em}
% \setlist{topsep=0.2em, partopsep=0em, itemsep=0.1em, parsep=0em}

\graphicspath{{Images/}}

% \renewcommand{\arraystretch}{1.2}

% Title formatting
\renewcommand{\maketitle}{
    \begin{center}
        \LARGE \textbf{ENGINEERING TRIPOS PART IIA} \\[0.5em]
        \Large \textbf{SF4 - DataLogger} \\[0.5em]
        \textbf{Final Report} \\[1.5em]
        \vspace{-1em}
        \small Will Hewes - wh365 \\ 
        Pembroke College \\ 
        \vspace{0.5em}
    \end{center}
}

\begin{document}
\pagenumbering{gobble}
% \includepdf[pages=-]{Handouts/IIA_Project_Coversheet Final Report.pdf}
\maketitle
\hrule
\tableofcontents
\newpage
\pagenumbering{arabic} \setcounter{page}{1}

\section{Introduction}
\label{sec:introduction}

The aim of this project was to develop a microcontroller-based 
automatic plant watering system.
The system can autonomously monitor soil moisture levels, 
plot the data over time, and provide options to water the plant
manually when required or in response to threshold moisture levels.
This allows for effective monitoring and care with minimal user intervention.

In addition to moisture sensing, the system also tracks temperature, 
and was supposed to track humidity and light -
though this could not be implemented due to component delays.
This means most of the key factors affecting plant health
can be closely tracked, enabling detailed analysis of
optimum conditions for plant health.
These additional features enhance the system's utility
for control and research.

The motivation behind creating this autonomous watering system was twofold. 
Firstly, it can be used as demonstrated on a small scale,
allowing direct control over the conditions of one 
or a small number of houseplants,
offering a convenient way to care for plants.
This will remove the majority of the care required
to look after these often frail plants,
which is particularly useful over holidays or 
during the hot Summer months.
As a university student with several well-loved plants,
this has immediate personal appeal.

On a larger scale, the system provides a modular, 
low-cost framework adaptable to industrial or agricultural applications, 
where automated irrigation and environmental monitoring are increasingly valuable.
The low cost and simple design of the system will be appealing 
to large scale, versatile integration,
and the ease with which components can be added
and the GUI customised will allow for rapid expansion into the sector.

\section{System Summary}
\label{sec:summary}

The system developed in this project provides a robust 
and modular platform for automated plant care, 
centred around real-time environmental monitoring and water delivery. 
It measures soil moisture and ambient temperature via analogue sensors 
connected to a microcontroller, and transmits this information via 
serial connection to a PC for logging and visualisation. 
In response to the measured conditions, 
or through direct user intervention, the system can 
actuate a servo-driven pinch valve that delivers 
a controlled quantity of water from an attached reservoir.

Operation can be either automatic -
based on configurable moisture thresholds -
or manual, with the user issuing commands through a graphical user interface (GUI). 
Sensor data is plotted in real-time and stored in a CSV file,
which will be automatically cleaned after use. 
The complete feedback loop between sensing, user interaction, 
and actuation allows for precise, low-effort maintenance of plant health.

The system is designed for domestic or small-scale use, 
particularly by individuals seeking a low-maintenance way to 
ensure plant health during travel or hot weather. 
However, the modular nature of the hardware and software makes it easily extensible. 
Additional sensors, wireless communication modules, 
or remote control features could be integrated with minimal disruption, 
paving the way for deployment in larger-scale agricultural or research settings.

\section{Project Management}
\label{sec:project_management}

\section{System Architecture}
\label{sec:architecture}
\subsection{Block Diagram}
\label{sec:block_diagram}
\subsection{Circuit Design}
\label{sec:circuit_design}
\subsection{Watering Mechanism}
\label{sec:water_system}

\section{Initial Development}
\label{sec:initial_development}
\subsection{Sensor Prototyping}
\label{sec:sensor_prototyping}
\subsection{GUI}
\label{sec:gui_simulation}
\subsection{Communication Protocol}
\label{sec:comm_protocol}

\section{Final Implementation}
\label{sec:final_implementation}
\subsection{Firmware}
\label{sec:firmware}
\subsubsection{Main Loop}
\label{subsec:main_loop}
\subsubsection{Signal Processing}
\label{subsec:firmware_processing}
\subsection{Software}
\label{sec:software}
\subsubsection{Module Structure}
\label{subsec:software_modules}
\subsubsection{User Interface}
\label{subsec:software_ui}

\section{Technical Challenges}
\label{sec:technical_problems}

\section{Further Improvements}
\label{sec:further_improvements}

\section{Conclusion}
\label{sec:conclusion}


\newpage
\appendix
\begin{thebibliography}{9}

\bibitem{arduino_servo}
Arduino. \textit{Servo Motor Basics with Arduino} : \\
\url{https://docs.arduino.cc/learn/electronics/servo-motors/}

\bibitem{tmp36}
Analog Devices. \textit{TMP35/TMP36/TMP37 Data Sheet} : \\
\url{https://www.analog.com/en/products/tmp36.html} 

\bibitem{arduino_tmp36}
ArduinoGetStarted. \textit{Arduino - TMP36 Temperature Sensor} : \\
\url{https://arduinogetstarted.com/tutorials/arduino-tmp36-temperature-sensor}

\bibitem{dfrobot}
DFRobot. \textit{Capacitive Soil Moisture Sensor SKU SEN0193} : \\
\url{https://wiki.dfrobot.com/Capacitive_Soil_Moisture_Sensor_SKU_SEN0193}

\bibitem{pinch_valve_design}
Printables. \textit{Pinch Valve Powered by Servo} : \\
\url{https://www.printables.com/model/247744-pinch-valve-powered-by-servo/files}

\end{thebibliography}

\section{Interim Report}
% \includepdf[pages=-]{Reports/First Interim Report.pdf} % e.g. [pages={1,3-5,7}] to include pages 1,3,4,5,7
% Featuring the Interim Report

\end{document}