\documentclass[a4paper,11pt]{article}
% Use ctrl + alt + V to view live pdf

% Packages
\usepackage[utf8]{inputenc} % For encoding
\usepackage[T1]{fontenc} % Better handling of accented characters and hyphenation
\usepackage{microtype} % Improves spacing and justification
\usepackage{amsmath, amssymb} % For equations and symbols
\usepackage{graphicx} % For including graphics/images
\usepackage{caption} % For customizing figure and table captions
\usepackage{subcaption} % For subfigures and subcaptions
\usepackage{float} % For fixing figure and table positions
\usepackage{booktabs} % For professional-looking tables
\usepackage{siunitx} % For consistent typesetting of units and numbers
\usepackage[margin=2cm]{geometry} % Adjusts page margins
\usepackage{fancyhdr} % For custom headers and footers
\usepackage{lmodern} % For a professional-looking font (main body font)
\usepackage{titlesec} % For title customization
\usepackage{array} % For custom table formatting
\usepackage[colorlinks=true, linkcolor=black, urlcolor=black]{hyperref} % Colored links without boxes
\usepackage{cleveref} % For improved cross-referencing    
\usepackage{multirow}
\usepackage{enumitem}
\usepackage{listings}
\usepackage{xcolor}
\usepackage{textcomp}
\usepackage{tabularx}
\usepackage{changepage}
\usepackage{tikz}
\usetikzlibrary{shapes.geometric, arrows}
% --- C++ Style ---
\lstdefinestyle{cpp-style}{
    language=C++,
    basicstyle=\ttfamily\footnotesize,
    keywordstyle=\color{blue}\bfseries,
    stringstyle=\color{orange},
    commentstyle=\color{gray}\itshape,
    numbers=left,
    numberstyle=\tiny\color{gray},
    numbersep=10pt,
    backgroundcolor=\color{white},
    showspaces=false,
    showstringspaces=false,
    breaklines=true,
    breakatwhitespace=true,
    tabsize=4,
    captionpos=b,
    frame=single,
    rulecolor=\color{black},
}

% --- Python Style ---
\lstdefinestyle{python-style}{
    language=Python,
    basicstyle=\ttfamily\footnotesize,
    keywordstyle=\color{blue}\bfseries,
    commentstyle=\color{gray}\itshape,
    stringstyle=\color{green!50!black},
    frame=single,
    breaklines=true,
    showstringspaces=false,
    captionpos=b
}
\renewcommand{\lstlistingname}{Code}
% Custom settings
\pagestyle{fancy}
\fancyhf{}
\fancyhead[L]{\textit{SF4 - DataLogger}} % Header left
\fancyhead[R]{\textit{cf573 | wh365}} % Header right 
\fancyfoot[C]{\thepage} % Footer center
\setlength{\headheight}{15pt} % Header height
\setlength{\parindent}{0em} % Indentation for paragraphs
\setlength{\parskip}{0.5em} % Add spacing between paragraphs
\setlength{\abovedisplayskip}{1em}
\setlength{\belowdisplayskip}{1em}
\setlength{\abovedisplayshortskip}{1em}
\setlength{\belowdisplayshortskip}{1em}
\setlist{topsep=0.2em, partopsep=0em, itemsep=0.1em, parsep=0em}

\graphicspath{{Images/Legacy/}}

% \renewcommand{\arraystretch}{1.2}

% Title formatting
\renewcommand{\maketitle}{
    \begin{center}
        \LARGE \textbf{ENGINEERING TRIPOS PART IIA} \\[0.5em]
        \Large \textbf{SF4 – DataLogger} \\[0.5em]
        \textbf{First Interim Report} \\[1.5em]
        \begin{tabularx}{0.7\textwidth}{X X}
            \centering \large Cheng Fang -- cf573 \\ \large Robinson College &
            \centering \large Will Hewes -- wh365 \\ \large Pembroke College
        \end{tabularx}
        \vspace{1em}
    \end{center}
}

\begin{document}
\maketitle
\hrule
\tableofcontents
\newpage

\section{Introduction}
\label{sec:Introduction}

This project aims to develop a microcontroller-based 
automatic plant watering system.
The system will autonomously monitor soil moisture levels, 
plot the data over time, and provide options to water the plant
at scheduled intervals or in response to threshold moisture levels.
This allows for effective monitoring and care with minimal user intervention.

In addition to moisture sensing, the system could be expanded to 
incorporate light and temperature sensors,
enabling further environmental data to be plotted and analysed.
These additional features enhance the system's utility
for controlling the conditions affecting plant health.

Cheng will be taking the lead on the circuit design and firmware,
and Will will be taking the lead on the software and hardware.

\section{Parts List}
\label{sec:Parts_List}

Table \ref{tab:component_order} in the Appendix displays the components 
ordered to supplement the project so far. Their usages are discussed further in sections 
\ref{sec:Analogue_Circuit_Design} and \ref{sec:Block_Diagram}. 

The implementation of the water valve in this project is based on an open-source design available online \cite{pinch_valve_design}, which utilizes a servo-powered pinch valve mechanism.

\section{Analogue Circuit Design}
\label{sec:Analogue_Circuit_Design}

Figure \ref{fig:Analogue_Circuit_Diagram_for_the_automatic_watering_system} 
in the Appendix shows the analogue circuit design for the automatic watering system.

The 5V wire acts as a supply rail, connecting to each of the components and routing the ground through where necessary, while the I/O pins map to their corresponding components. 
The 5V supply voltage lies within the operating range of all connected sensors, enabling direct connection to the Arduino without the need for additional filters or amplifiers, 
as the sensor outputs fall within the Arduino’s ADC input range and provide sufficient signal strength and stability~\cite{arduino_servo}.
Capacitors are added in parallel with each of the sensors to reduce sensitivity to noise, allowing the data to be tracked more accurately and eliminating some of the interference.

In designing the analogue circuit, we referred to the official documentation~\cite{arduino_servo,tmp36,dfrobot} of each sensor and actuator, as well as guidelines provided on the Arduino platform. 
For example, the TMP36 temperature sensor datasheet recommends placing a 0.1\,\textmu F ceramic bypass capacitor close to the supply pin to suppress high-frequency interference and prevent DC output shifts due to radio frequency noise~\cite{tmp36}. 
Similarly, a 100\,\textmu F capacitor is added across the power supply to the servo motor, as recommended in Arduino documentation, to smooth out voltage fluctuations caused by sudden changes in current draw during operation~\cite{arduino_servo}.

Once the sensors are received and tested, additional analogue circuitry may be introduced depending on their real-world performance. For instance, 
if the soil moisture sensor exhibits limited precision or slow response time, measures such as adding a low-pass filter to stabilise fluctuating readings, 
using an operational amplifier to improve signal resolution, or incorporating a comparator circuit to implement threshold-based triggering could be considered~\cite{dfrobot,yt}. 
These enhancements would help ensure more reliable and responsive behaviour, particularly in applications requiring real-time monitoring or control.


\section{Block Diagram}
\label{sec:Block_Diagram}

Figure \ref{fig:Block_Diagram_for_the_automatic_watering_system}
shown in the Appendix displays the general mode of operation for the system.
The sensor modules send real-time information to the Arduino,
which processes and transmits the data to the PC via USB serial communication.

This data is then stored by the PC and displayed graphically with Python, 
with an interactive GUI through which the 
theshold values, watering timings, or manual watering options
can be controlled by user input.

This is then fed back to the Arduino, 
which will either do nothing if water levels are above the threshold,
or dispense a controlled quantity of water by activating the servo
if moisture levels have fallen below the threshold
(or manual watering is toggled).

The Arduino will also monitor temperature levels and can be expanded to 
incorporate light sensors as well, allowing for increased datalogging capabilities.

\section{Firmware Design}

The firmware was developed in Arduino C/C++ based on the current circuit design. It is responsible for real-time sampling of sensor data, including the TMP36 temperature sensor and the capacitive soil moisture sensor. To improve robustness and reduce the effect of noise, the firmware performs averaging across multiple samples (10 samples by default) for each sensor before converting the raw ADC values into meaningful physical quantities—Celsius for temperature and a raw moisture scale from the soil sensor.

After processing, the firmware sends the results to the computer via serial communication every second in a simple, human-readable format (e.g., \texttt{T=24.5,M=312}). In addition, the firmware listens for control commands from the computer to drive the servo motor, which actuates the watering mechanism. The servo angle can be controlled by commands such as \texttt{SERVO=90}.

The firmware is initially based on and modified from example codes provided by the official documentation for each sensor and actuator:
\begin{itemize}
    \item TMP36 temperature sensor code was adapted from Analog Devices’ official documentation \cite{tmp36}.
    \item Soil moisture sensor code is adapted from DFRobot’s official Arduino example \cite{dfrobot}.
    \item Servo motor control follows the standard Arduino Servo library example \cite{arduino_servo}.
\end{itemize}
All logic and structure have been customized to meet the needs of this system while maintaining clarity and modularity for future expansion, such as adding digital filters or handling more sensors.

\section{Software Design}
\label{sec:Software_Design}

So far, the software design is in its early stages,
with a mock-up demonstration shown in the Appendix.
The final software implementation will feature 
Arduino communication to the PC via the serial,
plotting capabilities and a GUI,
and an Arduino controlled watering mechanism.

The GUI will be implemented through the Tkinter module in Python,
and data will be stored in a .csv file, allowing relevant data to be plotted.

\section{Conclusion}
\label{sec:Conclusion}

The project will work towards building an automated watering system,
capable of logging data with regard to soil moisture levels and temperature.
The key components have been ordered, 
and a preliminary block diagram and circuit design have been constructed - 
though they are subject to change as we test the component properties.

A mock-up software design has been created which allows us to 
simulate an Arduino serial communication and display the current values.
This will be further progressed by implementing 
Arduino communication once the sensors are implemented,
and by enabling plotting, data storage capabilities, and an interactive GUI.

\newpage
\appendix
\section{Parts List}

\begin{table}[H]
    \centering
    % \renewcommand{\arraystretch}{1.5} 
    % Save space (remove if not limiting)
    \makebox[\linewidth][c]{
    \resizebox{1.1\textwidth}{!}{
    \begin{tabular}{|c|c|c|c|c|}
        \hline
        \textbf{Order Code} & \textbf{Description of Component} & 
        \textbf{Qty} & \textbf{Unit Price (£)} & \textbf{Total Price (£)} \\
        \hline
        2946124 & Capacitive Soil Moisture Sensor Module & 1 & 4.69 & 4.69 \\
        \hline
        SC21096 & Mini servo & 1 & 2.94 & 2.94 \\
        \hline
        4030054 & Temperature sensor & 1 & 1.38 & 1.38 \\
        \hline
    \end{tabular}
    }   
    }
    \caption{Component Order Summary}
    \label{tab:component_order}
\end{table}

\section{Analogue Circuit Design}
\begin{figure}[H]
    \centering
    \includegraphics[width=0.6\textwidth]{Analogue Circuit Diagram.png}
    \caption{Analogue Circuit Diagram for the automatic watering system}
    \label{fig:Analogue_Circuit_Diagram_for_the_automatic_watering_system}
\end{figure}

\section{Block Diagram}
\begin{figure}[H]
    \centering
    \includegraphics[width=0.5\textwidth]{DataLogger Block Diagram2.png}
    \caption{Block Diagram for the automatic watering system}
    \label{fig:Block_Diagram_for_the_automatic_watering_system}
\end{figure}


\section{Firmware Code}
\subsection{firmware.cpp}
\lstinputlisting[
    style=cpp-style,
    caption={Arduino code for sending data to the computer end},
    label={Code:firmware.cpp}
]{firmware.cpp}
\section{Software Code}

This software code is just a placeholder to simulate
communication between the Arduino and the PC
before the real system is functional.

\subsection{sender.py}

\lstinputlisting[
    style=python-style,
    caption={Python script for sending data to the serial port},
    label={Code:sender.py}
]{Python/Legacy/sender.py}

\subsection{receiver.py}

\lstinputlisting[
    style=python-style,
    caption={Python script for receiving sending data to the serial port},
    label={Code:receiver.py}
]{Python/Legacy/receiver.py}

\begin{thebibliography}{9}

\bibitem{arduino_servo}
Arduino Documentation. \textit{servo in Arduino}. Available at: \url{https://docs.arduino.cc/learn/electronics/servo-motors/}

\bibitem{tmp36}
Analog Devices. \textit{TMP35/TMP36/TMP37 Data Sheet}. Available at: \url{https://www.analog.com/media/en/technical-documentation/data-sheets/TMP35_36_37.pdf}

\bibitem{arduino_tmp36}
Analog Devices. \textit{TMP36 with Arduino}. Available at: \url{https://arduinogetstarted.com/tutorials/arduino-tmp36-temperature-sensor}

\bibitem{dfrobot}
DFRobot. \textit{Capacitive Soil Moisture Sensor SKU:SEN0193}. Available at: \url{https://wiki.dfrobot.com/Capacitive_Soil_Moisture_Sensor_SKU_SEN0193}

\bibitem{yt}
YouTube. \textit{Improving Capacitive Soil Moisture Sensor Readings}, Available at: \url{https://youtu.be/QGCrtXf8YSs}

\bibitem{pinch_valve_design}
Printables. \textit{Pinch Valve Powered by Servo}. Available at: \url{https://www.printables.com/model/247744-pinch-valve-powered-by-servo/files}


\end{thebibliography}

\end{document}