\documentclass[a4paper,11pt]{article}
% Use ctrl + alt + V to view live pdf

% Packages
\usepackage[utf8]{inputenc} % For encoding
\usepackage[T1]{fontenc} % Better handling of accented characters and hyphenation
\usepackage{microtype} % Improves spacing and justification
\usepackage{amsmath, amssymb} % For equations and symbols
\usepackage{graphicx} % For including graphics/images
\usepackage{caption} % For customizing figure and table captions
\usepackage{subcaption} % For subfigures and subcaptions
\usepackage{float} % For fixing figure and table positions
\usepackage{booktabs} % For professional-looking tables
\usepackage{siunitx} % For consistent typesetting of units and numbers
\usepackage[top=1in, bottom=1in, left=1in, right=1in]{geometry} % Adjusts page margins
\usepackage{fancyhdr} % For custom headers and footers
\usepackage{lmodern} % For a professional-looking font (main body font)
\usepackage{titlesec} % For title customization
\usepackage{array} % For custom table formatting
\usepackage[colorlinks=true, linkcolor=black, urlcolor=black]{hyperref} % Colored links without boxes
\usepackage{cleveref} % For improved cross-referencing    
\usepackage{multirow}
\usepackage{enumitem}
\usepackage{listings}
\usepackage{xcolor}
\usepackage{textcomp}
\usepackage{tabularx}
\usepackage{changepage}
\usepackage{tikz}
\usetikzlibrary{shapes.geometric, arrows}

\lstdefinestyle{vhdl-style}{
    language=VHDL,
    basicstyle=\ttfamily\footnotesize,
    keywordstyle=\bfseries\color{blue},
    commentstyle=\itshape\color{gray},
    stringstyle=\color{red},
    numbers=left,
    numberstyle=\tiny\color{gray},
    stepnumber=1,
    breaklines=true,
    showstringspaces=false,
    frame=single
}
\lstset{style=vhdl-style}
\lstset{captionpos=b}
\lstset{basicstyle=\ttfamily\scriptsize} 
\renewcommand{\lstlistingname}{Program}

% Custom settings
\pagestyle{fancy}
\fancyhf{}
\fancyhead[L]{\textit{SF4 - DataLogger}} % Header left
\fancyhead[R]{\textit{Will Hewes - wh365}} % Header right 
\fancyfoot[C]{\thepage} % Footer center
\setlength{\headheight}{15pt} % Header height
\setlength{\parindent}{0em} % Indentation for paragraphs
\setlength{\parskip}{0.5em} % Add spacing between paragraphs
\setlength{\abovedisplayskip}{1em}
\setlength{\belowdisplayskip}{1em}
\setlength{\abovedisplayshortskip}{1em}
\setlength{\belowdisplayshortskip}{1em}
% \setlist{topsep=0em, partopsep=0em, itemsep=0em, parsep=0em}

\graphicspath{{C:/Users/willi/My Drive/Engineering/Obsidian Vault/Uni/IIA Projects/DataLogger-IIA/Images}}

\renewcommand{\arraystretch}{1.2}

% Title formatting
\renewcommand{\maketitle}{
    \begin{center}
        \LARGE \textbf{ENGINEERING TRIPOS PART IIA} \\ 
        \vspace{0.5em}
        \Large \textbf{SF4 - DataLogger} \\ 
        \vspace{0.5em}
        \textbf{First Interim Report} \\
        \vspace{1em}
        \large Will Hewes - wh365 \\ 
        Pembroke College \\ 
        \vspace{0.5em}
    \end{center}
}

\begin{document}
\maketitle
\hrule
\tableofcontents
\newpage

\section{Introduction}
\label{sec:Introduction}

\section{Analogue Circuit Design}
\label{sec:Analogue_Circuit_Design}

\section{Parts List}
\label{sec:Parts_List}

Table \ref{tab:component_order} below displays the components 
ordered to supplement the project so far.
Their usages are discussed further in sections 
\ref{sec:Analogue_Circuit_Design} and \ref{sec:Block_Diagrams}.

\begin{table}[H]
    \centering
    \renewcommand{\arraystretch}{1.5}
    \makebox[\linewidth][c]{
    \resizebox{1.1\textwidth}{!}{
    \begin{tabular}{|c|c|c|c|c|}
        \hline
        \textbf{Order Code} & \textbf{Description of Component} & 
        \textbf{Qty} & \textbf{Unit Price (£)} & \textbf{Total Price (£)} \\
        \hline
        2946124 & Capacitive Soil Moisture Sensor Module & 1 & 4.69 & 4.69 \\
        \hline
        SC21096 & Mini servo & 1 & 2.94 & 2.94 \\
        \hline
        4030054 & Temperature sensor & 1 & 1.38 & 1.38 \\
        \hline
    \end{tabular}
    }   
    }
    \caption{Component Order Summary}
    \label{tab:component_order}
\end{table}

\section{Block Diagrams}
\label{sec:Block_Diagrams}

Figure \ref{fig:Block_Diagram_for_the_automatic_watering_system}
shown below displays the general operation mode for the system.
The sensor modules send real-time information to the Arduino,
which processes and transmits the data to the PC via USB serial communication.

This data is then stored by the PC and displayed graphically in MatLab(?), 
with an interactive GUI through which the 
theshold values, watering timings, or manual watering options
can be controlled by user input.

This is then fed back to the Arduino, 
which will either do nothing if water levels are above the threshold,
or dispense a controlled quantity of water by activating the servo
if moisture levels have fallen below the threshold.

\begin{figure}[H]
    \centering
    \includegraphics[width=0.6\textwidth]{DataLogger Block Diagram2.png}
    \caption{Block Diagram for the automatic watering system}
    \label{fig:Block_Diagram_for_the_automatic_watering_system}
\end{figure}

\section{Software Design}
\label{sec:Software_Design}

\section{Conclusion}
\label{sec:Conclusion}

\appendix
\section{Appendix}

\end{document}